
\documentclass[12pt]{article}
\usepackage[utf8]{inputenc}
\usepackage[T1]{fontenc}
\usepackage{amsmath}
\usepackage{amsfonts}
\usepackage{amssymb}
\usepackage{graphicx}
\usepackage{array}
\usepackage{multirow}
\usepackage{geometry}
\usepackage{float}
\usepackage{tabularray}
\usepackage{listings}
\usepackage{xcolor}

\geometry{legalpaper, margin=1.5cm}
\lstdefinestyle{sqlStyle}{
	language=SQL,
	basicstyle=\ttfamily\small,
	keywordstyle=\color{blue},
	stringstyle=\color{red},
	commentstyle=\color{green!50!black},
	showstringspaces=false,
	breaklines=true,
	breakatwhitespace=true,
	tabsize=4,
	numbers=left,
	numberstyle=\scriptsize\color{gray},
	frame=single,
	frameround=tttt,
	rulecolor=\color{black},
	backgroundcolor=\color{gray!10},
	captionpos=b
}


\begin{document}
	
	\begin{center}
		\Huge RAPORT
	\end{center}
	
	\begin{table}[h!]
		\centering
		\begin{tblr}{
				width = \linewidth,
				colspec = {Q[156]Q[156]Q[156]Q[156]Q[156]Q[156]},
				row{1} = {c},
				column{4} = {c},
				column{6} = {c},
				cell{1}{1} = {c=6}{0.936\linewidth},
				cell{2}{2} = {c=5}{0.803\linewidth},
				cell{3}{2} = {c=5}{0.803\linewidth},
				cell{4}{2} = {c},
				cell{5}{2} = {c=5}{0.803\linewidth},
				hline{1,6} = {1}{-}{leftpos = 1, rightpos = 1},
				hline{1,6} = {2}{-}{leftpos = 1, rightpos = 1},
				hline{2,2} = {1}{-}{leftpos = 1, rightpos = 1},
				hline{2,2} = {2}{-}{leftpos = 1, rightpos = 1},
				vline{1,1} = {1}{-}{abovepos = 1, belowpos = 1},
				vline{1,1} = {2}{-}{abovepos = 1, belowpos = 1},
				vline{7,1} = {1}{-}{abovepos = 1, belowpos = 1},
				vline{7,1} = {2}{-}{abovepos = 1, belowpos = 1},
				hlines,
				vlines,
			}
			{AKADEMIA NAUK STOSOWANYCH W NOWYM SĄCZU\\Wydział Nauk Inżynieryjnych, Katedra informatyki} &  &  &  &  &  \\
			Przedmiot:  & Bazy danych          &  &  &  &  \\
			Temat:      & Aplikacja z bazą danych MySql - Lista zakupów - "Shopping dashboard"                       &  &  &  &  \\
			Grupa:      & IS-2(s)P2 & Data: & 30.05.2023 \\
			Osoby:      & Justyna Jachowicz
			&  &  &  &
		\end{tblr}
	\end{table}
	
	
	\section{Wykonane zadania}
	\begin{enumerate}
		\item Dodanie możliwości rejestracji użytkownika
		\item Dodanie możliwości logowania użytkowników
		\item Dodanie ról
		\item Aktualizacja widoków
		\item Implementacja widoczności przycisków do edycji i usuwania w zależności od roli użytkownika
		\item Aktualizacja skryptu SQL dodającego dane podczas startu aplikacji o role i podstawowych użytkowników 
		\begin{lstlisting}[style=sqlStyle, caption=data.sql, label=lst:sqlCode]
INSERT IGNORE INTO shopping_dashboard.roles(id, name) VALUES (1, 'ROLE_ADMIN');
INSERT IGNORE INTO shopping_dashboard.roles(id, name) VALUES (2, 'ROLE_USER');

INSERT IGNORE INTO shopping_dashboard.users (email, name, password) VALUES('admin@gmail.com', 'ADMIN', '$2a$10$bYbrEoMdvkh/ToEXBbn6p.C0tiRBz/iqy4HZaG1/8gtif23VA4peK');
INSERT IGNORE INTO shopping_dashboard.users (email, name, password) VALUES('justyna@gmail.com', 'Justyna', '$2a$10$bYbrEoMdvkh/ToEXBbn6p.C0tiRBz/iqy4HZaG1/8gtif23VA4peK');

INSERT IGNORE INTO shopping_dashboard.users_roles (user_id, role_id) VALUES(1, 1);
INSERT IGNORE INTO shopping_dashboard.users_roles (user_id, role_id) VALUES(2, 2);
		\end{lstlisting}

		
	\end{enumerate}
\section{Dane aplikacyjne}
\begin{enumerate}
	\item Użytkownik z rolą administracyjną - admin@gmail.com, 123456
	\item Użytkownik z rolą kliencką - justyna@gmail.com, 123456
\end{enumerate}
	\section{Poświęcony czas}
	\begin{itemize}
		\item Łącznie 9 godzin od ostatniego raportu
	\end{itemize}
	\section{Zadania na kolejny tydzień}
	\begin{enumerate}
		\item Dodanie możliwości edytowania, usuwania i dodawania produktów przez admina
		\item Zapisywanie produktów w listach zakupów dla danego użytkownika
	\end{enumerate}

\end{document}